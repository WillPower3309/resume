%% If you need to pass whatever options to xcolor
\PassOptionsToPackage{dvipsnames}{xcolor}

%% If you are using \orcid or academicons
%% icons, make sure you have the academicons 
%% option here, and compile with XeLaTeX
%% or LuaLaTeX.
% \documentclass[10pt,a4paper,academicons]{altacv}

%% Use the "normalphoto" option if you want a normal photo instead of cropped to a circle
% \documentclass[10pt,a4paper,normalphoto]{altacv}

\documentclass[10pt,a4paper]{altacv}
%% AltaCV uses the fontawesome and academicon fonts
%% and packages. 
%% See texdoc.net/pkg/fontawecome and http://texdoc.net/pkg/academicons for full list of symbols.
%% 
%% Compile with LuaLaTeX for best results. If you
%% want to use XeLaTeX, you may need to install
%% Academicons.ttf in your operating system's font 
%% folder.

% Change the page layout if you need to
\geometry{left=1cm,right=9cm,marginparwidth=6.8cm,marginparsep=1.2cm,top=1.25cm,bottom=1.25cm,footskip=2\baselineskip}

% Change the font if you want to.

% If using pdflatex:
\usepackage[T1]{fontenc}
\usepackage[utf8]{inputenc}
\usepackage[default]{lato}

% If using xelatex or lualatex:
% \setmainfont{Lato}
% Change the colours if you want to
\definecolor{Mulberry}{HTML}{72243D}
\definecolor{SlateGrey}{HTML}{2E2E2E}
\definecolor{LightGrey}{HTML}{666666}
\colorlet{heading}{Sepia}
\colorlet{accent}{Mulberry}
\colorlet{emphasis}{SlateGrey}
\colorlet{body}{LightGrey}

% Change the bullets for itemize and rating marker
% for \cvskill if you want to
\renewcommand{\itemmarker}{{\small\textbullet}}

\usepackage[colorlinks]{hyperref}

\begin{document}

\name{William McKinnon}
\tagline{SOFTWARE DEVELOPMENT ENGINEER}
\personalinfo{
  \email{contact@willmckinnon.com}
  \phone{226-971-3543}
  \github{github.com/willpower3309}
  \linkedin{linkedin.com/in/william-mckinnon}
  \homepage{willmckinnon.com}
  %% You MUST add the academicons option to \documentclass, then compile with LuaLaTeX or XeLaTeX, if you want to use \orcid or other academicons commands.
%   \orcid{orcid.org/0000-0000-0000-0000}
}

\begin{fullwidth}
    \makecvheader
    Toronto based Software Development Engineer experienced in building highly available data-intensive applications across a variety of tech stacks. Seeking a role in an innovative and challenging environment, building world-changing products.
\end{fullwidth}

% EXPERIENCE
\cvsection[sidebar]{Experience}

\cvevent{Amazon Web Services}{Toronto, Ontario}{SDE II}{December 2024 -- Present}
\vspace{-0.25em}
\textbf{\color{accent}SDE I}
\hfill\small\faCalendar\hspace{0.5em}June 2022 -- December 2024
\normalsize
\medskip
\begin{itemize}
    \item Contributed to the development and successful launch of Amazon Aurora Limitless, a next-generation horizontally scalable database offering.
    \item Designed and implemented the database endpoint, a critical component that intelligently routes client connections to healthy database nodes, ensuring high availability and reliability.
    \item Integrated Aurora Limitless into the AWS system test framework, enabling automated validation of core functionality across distributed environments.
    \item Identified and resolved numerous bugs prior to launch through rigorous test automation, significantly improving system stability and accelerating delivery timelines.
\end{itemize}

\divider

\cvevent{Vidyard}{Kitchener, Ontario}{Software Developer Intern}{April 2021 -- December 2021}
\begin{itemize}
    \item Collaborated with the sales development team to improve the Vidyard website, the Chrome extension, and the video player, while ensuring quality and stability for end users.
    \item Developed and improved user-facing features using JavaScript and TypeScript within the ReactJS and VueJS frameworks, contributing to both new development and existing codebase maintenance.
    \item Built and maintained end-to-end (E2E) tests using Cypress, improving test coverage and reducing regression risk across key components of the Vidyard platform.
    \item Delivered reliable, high-quality features and strengthened product stability through robust E2E test suites, directly supporting a better customer experience and increased development confidence.
\end{itemize}

\divider

\cvevent{Cognitive Systems Corp.}{Waterloo, Ontario}{Software Developer Intern}{January 2020 -- August 2020}
\begin{itemize}
    \item Contributed to both the testing and iOS development teams for the company's flagship product, WifiMotion, with responsibilities spanning mobile app development, QA automation, and backend load testing.
    \item Developed new features and enhancements for the iOS application using Swift and xcode, collaborating with the app team to deliver a reliable user experience.
    \item Added automated test cases to the QA pipeline using Python and the Pytest framework, improving test coverage and regression detection.
    \item Designed and implemented a custom load testing service using the Locust framework to assess the performance limits of backend services, from scratch through deployment.
    \item Successfully performed multiple load tests, identified performance bottlenecks, and provided a fully documented and maintainable load testing solution. Delivered stable code contributions and QA tests, enhancing product quality and backend resilience.
\end{itemize}

\end{document}

